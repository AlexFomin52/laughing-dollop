\documentclass[]{article}
\usepackage{lmodern}
\usepackage{iftex}
\usepackage[utf8]{inputenc}
\usepackage[russian]{babel}
\usepackage{graphicx}
\usepackage{tabularx}
\usepackage{listings}
\usepackage{caption}
\usepackage{subcaption}
\usepackage{amsmath, amsthm, amssymb, amsfonts}
\makeatletter
\@ifundefined{KOMAClassName}{
  \IfFileExists{parskip.sty}{
    \usepackage{parskip}
  }{
    \setlength{\parindent}{0pt}
    \setlength{\parskip}{6pt plus 2pt minus 1pt}}
}{
  \KOMAoptions{parskip=half}}
\makeatother
\usepackage{calc}
\usepackage{etoolbox}
\makeatletter
\patchcmd\longtable{\par}{\if@noskipsec\mbox{}\fi\par}{}{}
\makeatother
\makeatletter
\def\maxwidth{\ifdim\Gin@nat@width>\linewidth\linewidth\else\Gin@nat@width\fi}
\def\maxheight{\ifdim\Gin@nat@height>\textheight\textheight\else\Gin@nat@height\fi}
\makeatother

\setkeys{Gin}{width=\maxwidth,height=\maxheight,keepaspectratio}
\makeatletter
\def\fps@figure{htbp}
\makeatother
\setlength{\emergencystretch}{3em}
\providecommand{\tightlist}{%
  \setlength{\itemsep}{0pt}\setlength{\parskip}{0pt}}
\setcounter{secnumdepth}{-\maxdimen} %
\ifLuaTeX
  \usepackage{selnolig}
\fi
\IfFileExists{bookmark.sty}{\usepackage{bookmark}}{\usepackage{hyperref}}
\IfFileExists{xurl.sty}{\usepackage{xurl}}{} 
\urlstyle{same} 
\hypersetup{
  hidelinks,
  pdfcreator={LaTeX via pandoc}}

\author{Алексей Фомин}
\date{}

\begin{document}

МИНИСТЕРСТВО НАУКИ И ВЫСШЕГО ОБРАЗОВАНИЯ РОССИЙСКОЙ ФЕДЕРАЦИИ

федеральное государственное автономное образовательное учреждение
высшего образования

«САНКТ-ПЕТЕРБУРГСКИЙ ГОСУДАРСТВЕННЫЙ УНИВЕРСИТЕТ\\
АЭРОКОСМИЧЕСКОГО ПРИБОРОСТРОЕНИЯ»

КАФЕДРА 25

КУРСОВАЯ РАБОТА (ПРОЕКТ)\\
ЗАЩИЩЕНА С ОЦЕНКОЙ

РУКОВОДИТЕЛЬ


доц., канд. техн. наук Е.М. Линский

ПОЯСНИТЕЛЬНАЯ ЗАПИСКА\\
К КУРСОВОЙ РАБОТЕ (ПРОЕКТУ)\strut


ЗАДАЧА О НАЗНАЧЕНИЯХ\\
по дисциплине: \\

Основы программирования

РАБОТУ ВЫПОЛНИЛ



СТУДЕНТ гр. № 2354

Фомин А.А.\\

подпись, дата \\



Санкт-Петербург 2024
\newpage
\hypertarget{Оглавление}{%
\section{Оглавление}\label{Оглавление}}

\protect\hyperlink{Постановка задачи}{Постановка
задачи
\protect\hyperlink{Постановка задачи}{3}}

\protect\hyperlink{Алгоритм}{Алгоритм
\protect\hyperlink{Алгоритм}{4}}

\protect\hyperlink{Алгоритм полного перебора}{1.1
Алгоритм полного перебора
\protect\hyperlink{Алгоритм полного перебоа}{4}}

\protect\hyperlink{ПошаговыйПП}{1.2
Пошаговый пример для алгоритма
\protect\hyperlink{ПошаговыйПП}{5}}

\protect\hyperlink{Жадный алгоритм}{2.1
Жадный алгоритм
\protect\hyperlink{Жадный алгоритм}{8}}

\protect\hyperlink{ПошаговыйЖА}{2.2
Пошаговый пример для алгоритма
\protect\hyperlink{ПошаговыйЖА}{10}}

\protect\hyperlink{Псевдокод}{3.1
Псевдокод
\protect\hyperlink{Псевдокод}{12}}

\protect\hyperlink{Сравнение алгоритмов}{4.1
Сравнение алгоритмов
\protect\hyperlink{Сравнение алгоритмов}{13}}

\protect\hyperlink{Инструкция}{Инструкция
пользователя
\protect\hyperlink{Инструкция}{13}}

\protect\hyperlink{Тестовые примеры}{Тестовые
примеры
\protect\hyperlink{Тестовые примеры}{16}}

\protect\hyperlink{Заключение}{Заключение
\protect\hyperlink{Заключение}{19}}

\protect\hyperlink{Список литературы}{Список
литературы
\protect\hyperlink{Список литературы}{19}}

\protect\hyperlink{Приложение}{ПРИЛОЖНИЕ
\protect\hyperlink{Приложение}{20}}
\newpage
\hypertarget{section}{%
\section{}\label{section}}

\hypertarget{Постановка задачи}{%
\section{Постановка
задачи}\label{Постановка задачи}}

Задача о назначениях состоит в отыскании назначения, имеющего
минимальную стоимость и при котором каждая работа выполняется некоторым
человеком.

Дополнительные требования к задаче:

\begin{itemize}
\item
  Имеется n работников и n работ.
\item
  Стоимость назначения i-ого работника на j-ую работу равна c\_\{i,j\}.
\item
  Вход и выход программы: матрица работа-работники.
\item
  Для решения данной задачи требуется написать два алгоритма: полного
  перебора и жадный.
\item
  Провести исследование и сравнить время и точность жадного алгоритма и
  полного перебора.
\end{itemize}

Для задания использовалась следующая литература: А. Кофман, Введение в
прикладную комбинаторику, Наука, 1975 г

Пример, показывающий, что задача имеет смысл (таблица 1):

\begin{table}[h]
    \centering
    \begin{tabular}{|c|c|c|c|}
    \hline
    & Работа 1 & Работа 2 & Работа 3 \\ \hline
    Работник 1 & 100 & 200 & 300 \\ \hline
    Работник 2 & 150 & 150 & 150 \\ \hline
    Работник 3 & 200 & 150 & 250 \\ \hline
    \end{tabular}
    \caption{}
    \label{tab:work_and_workers}
\end{table}

Из таблицы, очевидно, следует, что лучшим назначением будет: работник 1
на работу 1, работник 2 на работу 3, работник 3 на работу 2. Так как
общая стоимость будет минимальной и равна 400. При этом распределение,
например, работник 1 на работу 1, работник 2 на работу 2, работник 3 на
работу 3, будет стоить 500.
\newpage

\hypertarget{Алгоритм}{%
\section{\textbf{Алгоритм}}\label{Алгоритм}}

В решении данной задачи требуется построить два алгоритма: полного
перебора и жадный.

Двумерный вектор \textbf{cost} хранит всю матрицу работников-работ.

\hypertarget{Алгоритм полного перебора}{%
\subsection{\textbf{1.1 Алгоритм полного
перебора}}\label{Алгоритм полного перебора}}

Алгоритм полного перебора подразумевает рассмотрение абсолютно всех
вариантов назначений и выбор наилучшего.

После запуска функции, рассмотренной в приложении 1, в вектор
\emph{perm} записываются числа от 1 до n (рисунок 1). Этот вектор нужен
для рассмотрения всех расстановок. Переменная \emph{minCost} принимает
максимально возможное числовое значение. И так как требуется найти
минимальное, оно будет обновляться с каждым новым минимальным найденным
значением.


\begin{figure}
    \centering
    \includegraphics[width=1\linewidth]{image1.png}
    \caption{}
    \label{fig:enter-label}
\end{figure}
\newpage
Затем, проходя по циклу, рассматривающему все размещения (внешний цикл),
для каждого конкретного размещения создается переменная
\emph{currentCost} равная 0. Для каждого работника \emph{i} от 0-ого до
n-ого выбирается работа под номером, стоящим в размещении \emph{perm},
имеющим индекс \emph{i} (внутренний цикл). Сумма назначения записывается
в \emph{currentCost}. Если \emph{currentCost} меньше, чем \emph{minCost},
то в \emph{minCost} записывается значение \emph{currentCost} и текущее
размещение \emph{perm} записывается в общее назначение
\textbf{assignment}. (рисунок 2)

\begin{figure}
    \centering
    \includegraphics[width=1\linewidth]{image2.png}
    \caption{}
    \label{fig:enter-label}
\end{figure}


Асимптотическая сложность алгоритма полного перебора: \emph{O(n!)}

\hypertarget{\textbf{ПошаговыйПП}}{%
\subsubsection{\texorpdfstring{\textbf{1.2 Пошаговый пример для
алгоритма}}{1.2 Пошаговый пример для алгоритма}}\label{ПошаговыйПП}}

Дана матрица 3 на 3 (таблица 2)

Таблица 2

\begin{table}[h]
    \centering
    \begin{tabular}{|c|c|c|c|}
    \hline
    & Работа 1 & Работа 2 & Работа 3 \\ \hline
    Работник 1 & 10 & 19 & 5 \\ \hline
    Работник 2 & 10 & 18 & 2 \\ \hline
    Работник 3 & 13 & 16 & 3 \\ \hline
    \end{tabular}
    \caption{}
    \label{tab:work_and_workers}
\end{table}

Следовательно, n = 3, \emph{perm} = \{0, 1, 2\}. Программа заходит во
внешний цикл.

Самой первой расстановкой, которую рассмотрит полный перебор, будет (0,
1, 2).
\begin{quote}
Первая итерация внешнего цикла(0, 1, 2):
\begin{quote}
Первая итерация внутреннего цикла:


\begin{quote}
\emph{i} = 0

\emph{currentCost} = \emph{cost}{[}0{]}{[}\emph{perm}{[}0{]}{]} =
\emph{cost}{[}0{]}{[}0{]} = 10.
\end{quote}
Вторая итерация внутреннего цикла:
\begin{quote}
\emph{i} = 1

{currentCost} = {currentCost} +
{cost}{[}1{]}{[}{perm}{[}1{]}{]} = 10 +
{cost}{[}1{]}{[}1{]} = 10 + 18 = 28.
\end{quote}
Третья итерация внутреннего цикла:
\begin{quote}
\emph{i} = 2

\emph{\textit{currentCost}} = \emph{currentCost} +
\emph{cost}{[}2{]}{[}\emph{perm}{[}2{]}{]} = 28 + cost{[}2{]}{[}2{]} =
28 + 3 = 31.
\end{quote}

\emph{currentCost} \textless{} \emph{minCost} = true

\emph{minCost} = \emph{currentCost} = 31
\end{quote}
Вторая итерация внешнего цикла (0, 2, 1):
\begin{quote}
Первая итерация внутреннего цикла:

\begin{quote}
\emph{i} = 0

\emph{currentCost} = \emph{cost}{[}0{]}{[}\emph{perm}{[}0{]}{]} =
\emph{cost}{[}0{]}{[}0{]} = 10.
\end{quote}

Вторая итерация внутреннего цикла:
\begin{quote}
\emph{i} = 1

\emph{currentCost} = \emph{currentCost} +
\emph{cost}{[}1{]}{[}\emph{perm}{[}1{]}{]} = 10 +
\emph{cost}{[}1{]}{[}2{]} = 10 + 2 = 12.
\end{quote}
Третья итерация внутреннего цикла:
\begin{quote}
\emph{i} = 2

\emph{currentCost} = \emph{currentCost} +
\emph{cost}{[}2{]}{[}\emph{perm}{[}2{]}{]} = 12 +
\emph{cost}{[}2{]}{[}1{]} = 12 + 16 = 28.
\end{quote}

\emph{currentCost} \textless{} \emph{minCost} = true

\emph{minCost} = \emph{currentCost} = 28
\end{quote}
Третья итерация внешнего цикла (1, 0, 2):
\begin{quote}
Первая итерация внутреннего цикла:

\begin{quote}
\emph{i} = 0

\emph{currentCost} = \emph{cost}{[}0{]}{[}\emph{perm}{[}0{]}{]} =
\emph{cost}{[}0{]}{[}1{]} = 19.

\end{quote}
Вторая итерация внутреннего цикла:
\begin{quote}
\emph{i} = 1

\emph{currentCost} = \emph{currentCost} +
\emph{cost}{[}1{]}{[}\emph{perm}{[}1{]}{]} = 19 +
\emph{cost}{[}1{]}{[}0{]} = 19 + 10 = 29.
\end{quote}
Третья итерация внутреннего цикла:
\begin{quote}
i = 2

\emph{currentCost} = \emph{currentCost} +
\emph{cost}{[}2{]}{[}\emph{perm}{[}2{]}{]} = 12 +
\emph{cost}{[}2{]}{[}2{]} = 29 + 3 = 32.
\end{quote}
\emph{currentCost} \textless{} \emph{minCost} = false

\emph{minCost} = 28
\end{quote}
Четвертая итерация внешнего цикла (1, 2, 0):
\begin{quote}
Первая итерация внутреннего цикла:

\begin{quote}
\emph{i} = 0

\emph{currentCost} = \emph{cost}{[}0{]}{[}\emph{perm}{[}0{]}{]} =
\emph{cost}{[}0{]}{[}1{]} = 19.

\end{quote}
Вторая итерация внутреннего цикла:
\begin{quote}
\emph{i} = 1

\emph{currentCost} = \emph{currentCost} +
\emph{cost}{[}1{]}{[}\emph{perm}{[}1{]}{]} = 19 +
\emph{cost}{[}1{]}{[}2{]} = 19 + 2 = 21.
\end{quote}
Третья итерация внутреннего цикла:
\begin{quote}
\emph{i} = 2

\emph{currentCost} = \emph{currentCost} +
\emph{cost}{[}2{]}{[}\emph{perm}{[}2{]}{]} = 12 +
\emph{cost}{[}2{]}{[}0{]} = 21 + 13 = 34.
\end{quote}
\emph{currentCost} \textless{} \emph{minCost} = false

\emph{minCost} = 28
\end{quote}
Пятая итерация внешнего цикла (2, 0, 1):
\begin{quote}
Первая итерация внутреннего цикла:

\begin{quote}
\emph{i} = 0

\emph{currentCost} = \emph{cost}{[}0{]}{[}\emph{perm}{[}0{]}{]} =
\emph{cost}{[}0{]}{[}2{]} = 5.

\end{quote}
Вторая итерация внутреннего цикла:
\begin{quote}
\emph{i} = 1

\emph{currentCost} = \emph{currentCost} +
\emph{cost}{[}1{]}{[}\emph{perm}{[}1{]}{]} = 5 +
\emph{cost}{[}1{]}{[}0{]} = 5 + 10 = 15.
\end{quote}
Третья итерация внутреннего цикла:
\begin{quote}
\emph{i} = 2

\emph{currentCost} = \emph{currentCost} +
\emph{cost}{[}2{]}{[}\emph{perm}{[}2{]}{]} = 15 +
\emph{cost}{[}2{]}{[}1{]} = 15 + 16 = 31.
\end{quote}
\emph{currentCost} \textless{} \emph{minCost} = false

\emph{minCost} = 28
\end{quote}
Шестая итерация внешнего цикла (2, 1, 0):
\begin{quote}
Первая итерация внутреннего цикла:
\begin{quote}

\emph{i} = 0

\emph{currentCost} = \emph{cost}{[}0{]}{[}\emph{perm}{[}0{]}{]} =
\emph{cost}{[}0{]}{[}2{]} = 5.

\end{quote}
Вторая итерация внутреннего цикла:
\begin{quote}
\emph{i} = 1

\emph{currentCost} = \emph{currentCost} +
\emph{cost}{[}1{]}{[}\emph{perm}{[}1{]}{]} = 5 +
\emph{cost}{[}1{]}{[}1{]} = 5 + 18 = 23.
\end{quote}
Третья итерация внутреннего цикла:
\begin{quote}
\emph{i} = 2

\emph{currentCost} = \emph{currentCost} +
\emph{cost}{[}2{]}{[}\emph{perm}{[}2{]}{]} = 23 +
\emph{cost}{[}2{]}{[}0{]} = 23 + 13 = 36.
\end{quote}
\emph{currentCost} \textless{} \emph{minCost} = false

\emph{minCost} = 28
\end{quote}
\end{quote}
Итого \emph{minCost} = 28, и \textbf{assignment} = \{0, 2, 1\}.

\hypertarget{Жадный алгоритм}{%
\subsection{2.1 Жадный
алгоритм}\label{Жадный алгоритм}}

Жадный алгоритм нужен для того, чтобы сэкономить время работы программы,
при этом, не сильно потеряв точность. Реализацией жадного алгоритма
является выбор наилучшей работы для работников по очереди.

После запуска функции, показанной в приложении 2, создается булевой
вектор \textbf{assigned} размера n, для хранения номеров уже назначенных
работ (рисунок 3). И создается переменная \emph{totalCost}, хранящая
сумму назначений.


\begin{figure}
    \centering
    \includegraphics[width=1\linewidth]{image3.png}
    \caption{}
    \label{fig:enter-label}
\end{figure}

Запускается внешний цикл для работника \emph{i} от 0-ого до n-ого,
создаются переменные \emph{minCost} и \emph{jobIndex}. (Рисунок 4).
Затем во внутреннем цикле, проходящем по работам \emph{j} от 0-ой до
n-ой, и если на эту работу еще не назначен работник, и назначение на
\emph{j}-ую работу минимальное для \emph{i}-ого работника, то она
назначается ему. К \emph{totalCost} прибавляется это назначение.

\begin{figure}
    \centering
    \includegraphics[width=1\linewidth]{image4.png}
    \caption{}
    \label{fig:enter-label}
\end{figure}
\newpage

Асимптотическая сложность алгоритма полного перебора: O(\(n^{2}\)).

\hypertarget{ПошаговыйЖА}{%
\subsubsection{\texorpdfstring{\textbf{2.2 Пошаговый пример для
алгоритма}}{2.2 Пошаговый пример для алгоритма}}\label{ПошаговыйЖА}}

Дана матрица 3 на 3(таблица 2)

n = 3, \textbf{assigned} = \{false, false, false\}.

Программа заходит во внешний цикл:
\begin{quote}
Первая итерация внешнего цикла:
\begin{quote}
\emph{i} = 0

Первая итерация внутреннего цикла:
\begin{quote}
\emph{j} = 0

\textbf{assigned}{[}0{]} = false \&\& \emph{cost}{[}0{]}{[}0{]}
\textless{} \emph{minCost}
\end{quote}
\emph{minCost} = 10

\emph{jobIndex} = 0

Вторая итерация внутреннего цикла:
\begin{quote}

\emph{j} = 1


\textbf{assigned}{[}1{]} = false \&\& \emph{cost}{[}0{]}{[}1{]}
\textgreater{} \emph{minCost}
\end{quote}
\emph{minCost} = 10

\emph{jobIndex} = 0

Третья итерация внутреннего цикла:
\begin{quote}

\emph{j} = 2


\textbf{assigned}{[}2{]} = false \&\& \emph{cost}{[}0{]}{[}2{]}
\textless{} \emph{minCost}
\end{quote}
\emph{minCost} = 5

\emph{jobIndex} = 2
\end{quote}
\textbf{assignment}{[}0{]} = 2

\textbf{assigned}{[}2{]} = true

\emph{totalCost} = 0 + 5 = 5

Вторая итерация внешнего цикла:
\begin{quote}
\emph{i} = 1

Первая итерация внутреннего цикла:
\begin{quote}
\emph{j} = 0

\textbf{assigned}{[}0{]} = false \&\& \emph{cost}{[}1{]}{[}0{]}
\textless{} \emph{minCost}
\end{quote}
\emph{minCost} = 10

\emph{jobIndex} = 0

Вторая итерация внутреннего цикла:

\begin{quote}
\emph{j} = 1


\textbf{assigned}{[}1{]} = false \&\& \emph{cost}{[}1{]}{[}1{]}
\textgreater{} \emph{minCost}
\end{quote}
\emph{minCost} = 10

\emph{jobIndex} = 0

Третья итерация внутреннего цикла:
\begin{quote}
\emph{j} = 2


\textbf{assigned}{[}2{]} = true \&\& \emph{cost}{[}1{]}{[}2{]}
\textless{} \emph{minCost}
\end{quote}
\emph{minCost} = 10

\emph{jobIndex} = 0
\end{quote}
\textbf{assignment}{[}1{]} = 0

\textbf{assigned}{[}0{]} = true

\emph{totalCost} = 5 + 10 = 15

Третья итерация внешнего цикла:
\begin{quote}
\emph{i} = 2

Первая итерация внутреннего цикла:
\begin{quote}
\emph{j} = 0

\textbf{assigned}{[}0{]} =true \&\& \emph{cost}{[}2{]}{[}0{]}
\textless{} \emph{minCost}
\end{quote}
\emph{minCost} = {$\infty$}

\emph{jobIndex} = -1

Вторая итерация внутреннего цикла:
\begin{quote}
\emph{j} = 1


\textbf{assigned}{[}1{]} = false \&\& \emph{cost}{[}2{]}{[}1{]}
\textless{} minCost
\end{quote}
\emph{minCost} = 16

\emph{jobIndex} = 1

Третья итерация внутреннего цикла:

\begin{quote}
\emph{j} = 2


\textbf{assigned}{[}2{]} = true \&\& \emph{cost}{[}2{]}{[}2{]}
\textless{} \emph{minCost}
\end{quote}
\emph{minCost} = 16

\emph{jobIndex} = 1
\end{quote}
\textbf{assignment}{[}2{]} = 1

\textbf{assigned}{[}1{]} = true

\emph{totalCost} = 15 + 16 = 31
\end{quote}
Итого \emph{totalCost} = 31, \textbf{assignment} = \{2, 0, 1\}.

\hypertarget{Псевдокод}{%
\subsection{3.1
Псевдокод}\label{псевдокод}}

Задача о назначениях(n)

Вход: матрица n на n, с натуральными числами -- стоимостью назначения
работников на работы.

Выход: минимальное назначение полным перебором и жадным алгоритмом.

Полный перебор:

\begin{quote}
minCost = {$\infty$}

perm(n)

для каждой расстановки(perm(n)):
\begin{quote}
currentCost = 0

для каждого назначения работника(n):
\begin{quote}
currentCost += cost{[}i{]}{[}perm{[}i{]}{]}~~
\end{quote}
if (currentCost \textless{} minCost):
\begin{quote}
minCost = currentCost

assignment = perm
\end{quote}
\end{quote}
\end{quote}
Жадный алгоритм:

assigned(n, false)

\begin{quote}
totalCost = 0

для каждого работника(n):
\begin{quote}
minCost = {$\infty$}

jobIndex = -1;

для каждой работы(n):
\begin{quote}
if (!assigned{[}j{]} \&\& cost{[}i{]}{[}j{]} \textless{} minCost):
\begin{quote}
minCost = cost{[}i{]}{[}j{]}

jobIndex = j
\end{quote}
\end{quote}
assignment{[}i{]} = jobIndex

assigned{[}jobIndex{]} = true

totalCost += minCost
\end{quote}
\end{quote}
\hypertarget{Сравнение алгоритмов}{%
\subsection{4.1 Сравнение
алгоритмов}\label{Сравнение алгоритмов}}


\begin{table}[h]
    \centering
    \begin{tabular}{|c|c|c|}
    \hline
    Число n – кол-во работников & Полный перебор & Жадный алгоритм \\ \hline
    1 & 500 нс & 600 нс \\ \hline
    2 & 600 нс & 700 нс \\ \hline
    3 & 800 нс & 800 нс \\ \hline
    4 & 1100 нс & 1100 нс \\ \hline
    5 & 10500 нс & 1200 нс \\ \hline
    6 & 18500 нс & 1300 нс \\ \hline
    7 & 155200 нс & 1400 нс \\ \hline
    8 & 666600 нс & 1300 нс \\ \hline
    9 & 637900 нс & 1400 нс \\ \hline
    10 & 76570300 нс & 1400 нс \\ \hline
    11 & 761096900 нс & 1600 нс \\ \hline
    12 & 9060960400 нс & 1600 нс \\ \hline
    13 & 119710262400 нс & 1800 нс \\ \hline
    14 & 1854436035900 нс & 2000 нс \\ \hline
    \end{tabular}
    \caption{}
    \label{tab:comparison}
\end{table}
\newpage
В таблице 3 представлена зависимость времени выполнения алгоритмов от
числа n. График зависимости представлен на рисунке 5.

\begin{figure}
    \centering
    \includegraphics[width=1\linewidth]{image5.png}
    \caption{}
    \label{fig:enter-label}
\end{figure}
\newpage


Для установления точности жадного алгоритма был проведён эксперимент. Для 10 случайных матриц размером n на n посчитана потеря жадного алгоритма от лучшего назначения, посчитано их среднее значение. График зависимости точности жадного алгоритма от n представлен на рисунке 6.
\begin{figure}
    \centering
    \includegraphics[width=1\linewidth]{loss.pdf}
    \caption{}
    \label{fig:enter-label}
\end{figure}
\newpage
\hypertarget{Инструкция}{%
\section{Инструкция
пользователя}\label{Инструкция}}

Перед тем как запускать программы, пользователю необходимо создать
текстовый файл, например input.txt. Чтобы не пришлось указывать
расположение файла, его рекомендуется поместить в директорию программы,
чтобы в командной строке писать просто input.txt вместо, например,  
D:\textbackslash program\textbackslash course\textbackslash course\textbackslash input.txt.

В первой строке файла написано натуральное число n, затем идет матрица n
на n натуральных чисел. Каждой строке соответствует работник, каждой
работе соответствует работа.

Программа выполняется автоматически, и результат выводится в файл
output.txt в той же директории.

\hypertarget{Тестовые примеры}{%
\section{\texorpdfstring{\textbf{Тестовые
примеры}}{Тестовые примеры}}\label{Тестовые примеры}}

Пример 1:

\begin{quote}
input:
\begin{quote}
3

10 19 5

10 18 2

13 16 3
\end{quote}
output:
\begin{quote}
Brute force:

Minimal cost: 28

Workers arrangement: Worker 1 job 1

Worker 2 job 3

Worker 3 job 2

Greedy algorithm:

Minimal cost: 31

Assignments: Worker 1 job 3

Worker 2 job 1

Worker 3 job 2
\end{quote}
\end{quote}
Пример 2:
\begin{quote}
input:

\begin{quote}
6

10 16 19 9 19 15

9 15 17 16 13 12

11 10 10 10 9 8

12 14 13 11 10 15

9 9 9 17 17 19

20 9 17 20 8 15
\end{quote}
output:
\begin{quote}
Brute force:

Minimal cost: 54

Workers arrangement: Worker 1 job 4

Worker 2 job 1

Worker 3 job 6

Worker 4 job 5

Worker 5 job 3

Worker 6 job 2

Greedy algorithm:

Minimal cost: 62

Assignments: Worker 1 job 4

Worker 2 job 1

Worker 3 job 6

Worker 4 job 5

Worker 5 job 2

Worker 6 job 3
\end{quote}
\end{quote}
Пример 3:
\begin{quote}
input:
\begin{quote}
12


10 16 19 9 19 15 9 15 17 16 13 12

12 14 13 11 10 15 9 9 9 17 17 19

10 16 19 9 19 15 9 9 9 17 17 19

12 14 13 11 10 15 15 17 16 13 12 14

20 9 17 20 8 15 20 9 17 20 8 15

11 10 10 10 9 8 11 10 10 10 9 8

10 16 19 9 19 15 9 15 17 16 13 12

12 14 13 11 10 15 9 9 9 17 17 19

10 16 19 9 19 15 9 9 9 17 17 19

12 14 13 11 10 15 15 17 16 13 12 13

20 9 17 20 8 15 20 9 17 20 8 15

11 10 10 10 9 8 11 10 10 10 9 8
\end{quote}
output:
\begin{quote}
Brute force:

Minimal cost: 115

Workers arrangement: Worker 1 job 1

Worker 2 job 3

Worker 3 job 4

Worker 4 job 5

Worker 5 job 2

Worker 6 job 6

Worker 7 job 7

Worker 8 job 8

Worker 9 job 9

Worker 10 job 10

Worker 11 job 11

Worker 12 job 12

Greedy algorithm:

Minimal cost: 126

Assignments: Worker 1 job 4

Worker 2 job 7

Worker 3 job 8

Worker 4 job 5

Worker 5 job 11

Worker 6 job 6

Worker 7 job 1

Worker 8 job 9

Worker 9 job 2

Worker 10 job 3

Worker 11 job 12

Worker 12 job 10
\end{quote}
\end{quote}
\hypertarget{Заключение}{%
\section{Заключение}\label{Заключение}}

В данной работе были рассмотрены два алгоритма для решения задачи о
назначениях: алгоритм полного перебора и жадный алгоритм. Также было
выявлено, что алгоритм полного перебора, несмотря на свою точность,
работает очень долго, уже для матрицы 20 на 20 алгоритм будет работать
30 лет. Жадный же алгоритм работает быстро, но при этом теряет свою
точность.

Для более точного вычисления времени работы алгоритмов используется
библиотека chrono, благодаря которой время работы алгоритма можно
вычислять в наносекундах. График зависимости времени выполнения
алгоритмов от числа n сделан в логарифмическом масштабе для более
удобного представления.

\hypertarget{Список литературы}{%
\section{Список
литературы}\label{Список литературы}}

Кофман, А. Введение в прикладную комбинаторику / А. Кофман; Наука. --
М., 1975. -- 477 с.

Рейнгольд, Э. Комбинаторные алгоритмы / Э. Рейнгольд, Ю. Нивергельт, Н.
Део; Мир. -- М., 1980. -- 476 с.
\newpage
\hypertarget{Приложение}{%
\section{ПРИЛОЖНИЕ}\label{Приложение}}

\begin{figure}
    \centering
    \includegraphics[width=1\linewidth]{image6.png}
    Приложение 1 - Полный перебор
    \label{fig:enter-label}
\end{figure}


\begin{figure}
    \centering
    \includegraphics[width=1\linewidth]{image7.png}
    Приложение 2 - Жадный Алгоритм
    \label{fig:enter-label}
\end{figure}



\end{document}
